\documentclass[10pt]{report}
\usepackage[a4paper]{geometry}
\usepackage[myheadings]{fullpage}
\usepackage{fancyhdr}
\usepackage{lastpage}
\usepackage{graphicx, wrapfig, subcaption, setspace, booktabs}
\usepackage[T1]{fontenc}
\usepackage[font=small, labelfont=bf]{caption}
\usepackage{fourier}
\usepackage[protrusion=true, expansion=true]{microtype}
\usepackage[english]{babel}
\usepackage{sectsty}
\usepackage{url, lipsum}
\usepackage{tgbonum}
\usepackage{hyperref}
\usepackage{xcolor}

\newcommand{\HRule}[1]{\rule{\linewidth}{#1}}
\onehalfspacing
\setcounter{tocdepth}{5}
\setcounter{secnumdepth}{5}

\begin{document}
{\fontfamily{cmr}\selectfont
\title{ \normalsize \textsc{}
		\\ [2.0cm]
		\HRule{0.5pt} \\
		\LARGE \textbf{\uppercase{Deliverable 3}
		\\Software Engineering Process
		\HRule{2pt} \\ [0.5cm]
		\normalsize  \vspace*{5\baselineskip}}
		}
\author{
		Sidhant Gupta \\ 
		40059256\\  }
\date{August 2, 2019}
\maketitle
\newpage

\sectionfont{\scshape}

\section{Problem 5}
%about code review
Code review, also called peer review, is a software quality assurance activity that involves manual or automated review of the source code by software engineer other than the code's author. This section reviews the F3 : sinh(x) function. 
\subsection{Review Approach}
My review approach is based on the following criteria :

\begin{itemize}
    \item Code Style
    \item Naming Conventions 
    \item Indentations
    \item Readability
    \item Design
    \item Maintainability 
\end{itemize}
'PMD' was used as a static source code review tool. 



\subsection{Review Results}
Class: SinHCalculator
\begin{enumerate}
    \item lines [15] : variable 'k' not named properly. The purpose of the variable is not clear
    \item lines [23-30]: The two for loops can be combined into one as they operate over the same range. 
    \item lines[35]: Spelling error in comments. "Stoping" should be "stopping"
    \item lines [3,9] : Class should have a private constructor, making it a utility class.
\end{enumerate}
Class: ValueNotSupportedException
\begin{enumerate}
    \item lines[2-7]: No Javadoc comments present
\end{enumerate}
Class: Calculator
\begin{enumerate}
    \item lines[35]: Closing of scanner should be in a 'finally' block
    \item lines[42]: Spelling error in comments. "tob e" should be "to be"
\end{enumerate}


\subsection{Comments}
\begin{enumerate}
    \item The code has good readability and is easy to understand
    \item The code is maintainable and hence it easy to test and make changes to it
    \item The code lacks the use of a proper design pattern or a framework which makes it uncertain about how to add new functionality to this project
    \item The code is well documented and proper naming conventions have been followed.
\end{enumerate}

\newpage
\section{Problem 7}
This section cover the testing of the function F4: $ \log_b (x) $ .
\\
The JUnit 4 framework has been used for writing and verifying the unit tests of the given function.
The given test cases all map to the functional requirements and the test cases executed successfully. 

\section*{Repository}
GitHub URL : https://github.com/SidhantGupta92/soen-6011-40059256/
\section*{References}
[1]. [Wikipedia] URL: https://en.wikipedia.org/wiki/Code\_review
\newline
[2]. [PMD] URL: https://github.com/pmd
\newline
[3]. [blog.jetbrains.com] URL: https://blog.jetbrains.com/upsource/2015/07/23/what-to-look-for-in-a-code-review/

}
\end{document}