\documentclass[12pt]{article}
\usepackage[utf8]{inputenc}

\title{ \centering Software Engineering Process \\ Deliverable 1}

\author{Sidhant Gupta \\ SID : 40059256}

\date{19 July 2019}

\begin{document}

\maketitle
\newpage
\section{Problem-1}

The Tangent(tan) function is one of the most familiar trigonometric functions. 
\newline
In case of a right angled triangle, the tan of an angle can be defined as: the ratio between the length of the opposite side and the length of the adjacent side. This can be denoted as  :
\newline
\newline
$tan( \theta ) = \frac{length of opposite)}{length of adjacent }$
\newline
\newline
One of the common identities that shows the relation between various trigonometric identities is :
\newline
\newline
$tan(\theta) = \frac{ sin(\theta)}{cos(\theta) }$
\newline
\newline
Features of tan( ) function :
\newline
\begin{enumerate}
\item Domain : $ \{x\mid x \neq \frac{\pi}{2} + k\pi , k= ...,-1,0,1,...   \}$
\item Co-Domain : R
\item Period : $\pi $
\item tan(x) is symmetric 
\end{enumerate}
\newpage
\section{Problem-2}
Functional requirements:
\newline
1. Assumptions:
\begin{enumerate}
    \item All inputs to the function will be numeric
    \item Input will be a real number
    \item Input will be assumed in degrees 
\end{enumerate}
2. Requirements:
\newline
\begin{itemize}
\item ID : ETRN-REQ-1
    \begin{enumerate}
        \item Type : Functional requirement 
		\item Version : 1.0
		\item Difficulty : Easy
		\item Owner : Author
		\item Description : User can enter any valid number for the function tan(x)
        \item Rationale : The tan() function returns valid values for all inputs except only for $ x = \frac{\pi}{2} + k\pi $ . In this case, NaN can be returned.   
    \end{enumerate}
\item ID : ETRN-REQ-2
    \begin{enumerate}
        \item Type : Functional requirement 
		\item Version : 1.0
		\item Difficulty : Medium
		\item Owner : Author
		\item Description : tan(x) returns the computed value when $ x \neq \frac{\pi}{2} + k\pi$
        \item Rationale : The domain for tan(x) is satisfied. Hence calculated value will be returned.
    \end{enumerate}
    \newpage
\item ID : ETRN-REQ-3
    \begin{enumerate}
        \item Type : Functional requirement 
		\item Version : 1.0
		\item Difficulty : Easy
		\item Owner : Author
		\item Description : tan(x) returns NaN when $ x = \frac{\pi}{2} + k\pi$
        \item Rationale : The domain for tan(x) is not satisfied. The function returns NaN. No computation is required.

    \end{enumerate}
\item ID : ETRN-REQ-4
    \begin{enumerate}
        \item Type : Functional requirement 
		\item Version : 1.0
		\item Difficulty : Easy
		\item Owner : Author
		\item Description : When input is non-numeric, Incorrect format exception is thrown
        \item Rationale : tan(x) is a mathematical function that only accepts numerical values.

    \end{enumerate}
\end{itemize}
\newpage
\section{Problem-3}
\textbf{Algorithm 1 : Calculation of tan(x) using Taylor series expansion}
\begin{enumerate}
    \item if x not in range(0,180) then,
    \item \hspace*{10mm} Subtract largest multiple of 180 less than x 
    \item end if
    \item if $ x > 90 \hspace{1mm} \wedge \hspace{1mm} x < 180 $ then,
    \item \hspace*{10mm} $ x  \leftarrow x - 180 $
    \item end if
    \item if $ x > 45 \hspace{1mm} \wedge \hspace{1mm} x < 90 $ then,
    \item \hspace*{10mm} $x1 \leftarrow 90 - x $
    \item \hspace*{10mm} return $ \frac{1}{ tan(x1) }  $
    \item end if
    \item if $ x > 22.5 \hspace{1mm} \wedge \hspace{1mm} x < 45 $ then,
    \item \hspace{10mm} return $ \frac{ 2*tan(\frac{x}{2})}{1 - tan^2(\frac{x}{2})}$
    \item end if 
    \item if $ x < 22.5 $ then,
    \item \hspace{10mm} $ z \leftarrow x * \frac{\pi}{180} $
    \item \hspace{10mm} return $ z + \frac{z^3}{3} + \frac{2z^5}{15} + \frac{17z^7}{315} $
\end{enumerate}
\newpage
\textbf{Algorithm 2 : Computing tan(x) as a ratio of sin and cos}
\begin{enumerate}
    \item $x1 \leftarrow sin(x)$
    \item $x2 \leftarrow cos(x)$
    \item return $ \frac{x1}{x2}$
\end{enumerate}

\textbf{Factors in algorithm selection :}
\begin{enumerate}
    \item Algorithm 1 gives accuracy of $\pm 0.000006$
    \item Algorithm 1 is efficient 
    \item Algorithm 2 uses ratio of 2 separately calculated values. This approach lacks precision.
    \item Algorithm 2 is dependent on sin(x) and cos(x) functions
    
    
\end{enumerate}
\end{document}

